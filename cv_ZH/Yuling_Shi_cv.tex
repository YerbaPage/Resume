\documentclass[a4paper]{article}
\usepackage{hyperref}
\usepackage[UTF8]{ctex}
\linespread{1.175}
\usepackage[top=0.5in, bottom=0.5in, left=0.5in, right=0.5in]{geometry}
\usepackage{enumitem}

\begin{document}
\begin{center}
    \thispagestyle{empty}
    \large \textbf{\kaishu{石雨凌} \\}
    \normalsize 邮箱: shiyuling@163.sufe.edu.cn $\mid$ 电话: (86) 187-7023-2576 $\mid$ \href{https://github.com/YerbaPage}{GitHub: github.com/YerbaPage}   \\
    % Address: No.777 Guoding Road, Yangpu District, Shanghai, P.R.China 200433\\
    \hrulefill
\end{center}
\noindent \textbf{\underline{\kaishu{教育背景}}} \\
\noindent \textbf{上海财经大学} \hfill 中国, 上海 \\
\textit{数学学院, 财经数学实验班} \hfill 2018.9 - $\quad$
\begin{itemize}[noitemsep,nolistsep,leftmargin=*]
    \item {所学专业: 数学与应用数学, 年级排名 \textbf{13/104}, 本学期年级排名 \textbf{1/104}.}
    \item {主要课程: \textbf{数学分析 (3.3), 高等代数 (4.0), 概率论 (4.0), 文本挖掘 (4.0), 深度学习 (4.0), 人工智能 (4.0).}}
    \item {奖学金: 人民奖学金 \textbf{(10$\%$)}, 单项奖学金 \textbf{(2$\%$)}, 特色之星 \textbf{(2$\%$)}.}
    \item {熟练掌握: Python (深度学习, 计算数学), MATLAB, Bash, \LaTeX, Git; 具备经验: C, Make, Javascript, SQL.\\}
          % FORTAN CUDA maybe
          % \item {Team working: Caption of school table tennis team, president of school table tennis club.\\}
          % \item {Solid expertise in: Python (Pytorch, Tensorflow, FEM), MATLAB, Bash, \LaTeX, Git; experienced in: C, C\texttt{++}, CMake, Javascript, SQL, SPSS\\} % FORTAN CUDA maybe
\end{itemize}

% \textbf{University name} \hfill City, State \\
% \textit{Degree name + Specialization} \hfill GPA: x.x/x.x \hfill month-year \\

%%%%%%%%%%%%%%%%%%%%%%%%%%%%%%%%%%%%%%%%%%%%%%%%%%%%%%%%%%%%%%%
% WORK EXPERIENCE
% What did you do? -> Project goals OR what problem did you solve?
% How did you do it? -> Skills and technologies.
% What impact did you create? -> Numbers and percentages.
% Example: 
% + Developed an app for matching mentor and mentees for Android and iOS platform.
% + Successfully matched 85% of the applications and randomized the rest.
% 
% Talk about team work, initiative, soft skills.
%
% Can also include personal projects, competitions, contribution to Open source.
%%%%%%%%%%%%%%%%%%%%%%%%%%%%%%%%%%%%%%%%%%%%%%%%%%%%%%%%%%%%%%%

\noindent \textbf{\underline{\kaishu{发表论文}}} \\
\noindent [1] Xuehai Huang, \textbf{Yuling Shi}, Wenqing Wang:
A Morley-Wang-Xu element method for a fourth order elliptic singular perturbation problem. \textit{Journal of Scientific Computing} \textbf{(SCI 一区)}, 2021. \\ % arXiv:2011.14064

% \noindent \textbf{Company name} \hfill City Name, State \\
% \textit{Role name} \hfill Month, Year $-$ Month, Year
% \begin{itemize}[noitemsep,nolistsep,leftmargin=*]
% \item {Developed XYZ using XYZ that led to X\% improvement.}
% \item {... \\}
% \end{itemize}

% \noindent \textbf{Competition Name} \hfill City Name, State \\
% \textit{Role name, Team name} \hfill Month, Year $-$ Month, Year
% \begin{itemize}[noitemsep,nolistsep,leftmargin=*]
% \item {Developed XYZ using XYZ that led to X\% improvement.}
% \item {Came in the top 10 OR received the most innovative award.}
% \item {... \\}
% \end{itemize}

%%%%%%%%%%%%%%%%%%%%%%%%%%%%%%%%%%%%%%%%%%%%%%%%%%%%%%%%%%%%%%%
% PROJECT
% What did you do?
% How did you do it? -> Skills and technologies
% What impact did you create? -> Numbers and percentages.
%
% Talk about team work, initiative, soft skills.
%
% Can also include personal projects, competitions, contribution to Open source.
%%%%%%%%%%%%%%%%%%%%%%%%%%%%%%%%%%%%%%%%%%%%%%%%%%%%%%%%%%%%%%%
\noindent \textbf{\underline{\kaishu{研究经历}}} \\
\noindent \textbf{研究助理, 文本生成中的文段重复问题研究} \hfill CILVR 实验室, 纽约大学 (Remote)
\\ \textit{指导教师: Prof. He He} \hfill  2021.5 - $\quad$ 
\begin{itemize}[noitemsep,nolistsep,leftmargin=*]
    \item {主动联系参与的科研项目, 目前正通过大量文献阅读与实验, 探寻改善机器生成文本中文段重复出现问题的方案.\\}
\end{itemize}

\noindent \textbf{研究助理, 预训练模型的可解释性研究} \hfill 中国, 上海
\\ \textit{指导教师: Dr. Wanyun Cui} \hfill  2021.1 - 2021.4
\begin{itemize}[noitemsep,nolistsep,leftmargin=*]
    \item {通过模型可解释性相关的最新顶会论文与 workshop, 学习基于梯度和基于扰动的多种深度学习模型解释方法.}
    \item {从损失函数泰勒展开中一阶和二阶项的角度, 分析了 BERT 模型在训练过程中对文本词汇之间关系的理解情况.\\}
\end{itemize}

\noindent \textbf{研究助理, 四阶偏微分方程的高效有限元方法研究} \hfill 中国, 上海
\\ \textit{指导教师: Prof. Xuehai Huang} \hfill 2020.6 - 2021.2
\begin{itemize}[noitemsep,nolistsep,leftmargin=*]
    \item {对于四阶奇异扰动方程, 通过对方程的解耦提出了一种使用最简单有限元的混合元方法, 并由此设计了快速求解算法.}
    \item {主动联系国外教授及博士后讨论问题, 发现开源软件包 \textit{scikit-fem} 中存在的的 bug 并贡献代码.}
    \item {最终论文已在计算数学知名杂志 \textit{Journal of Scientific Computing} \textbf{(SCI 一区)} 上发表. \\}
\end{itemize}

\noindent \textbf{项目负责人, 金融领域知识图谱的自动化构建项目}  \hfill 中国, 上海
\\ \textit{指导教师: Prof. Xuehai Huang} \hfill  2020.5 - 2020.12
\begin{itemize}[noitemsep,nolistsep,leftmargin=*]
    \item {带领团队与平安公司导师合作, 每周组织组会学习分享实体识别, 关系抽取等知识图谱构建的关键技术.}
    \item {完成了从数据的自动化爬取, 构建知识图谱到 Neo4j 图数据库存储和前端应用的完整流程.}
    \item {被选为``大学生创新创业计划''优秀项目, 并在学术论坛登台展示 \textbf{(2\%)}.\\}
\end{itemize}

\noindent \textbf{项目负责人, Kaggle ``Question Answering'' 竞赛项目} \hfill 中国, 上海
\\ \textit{指导教师: Prof. Hui Fang and Prof. Qi Deng} \hfill 2020.2 - 2020.6
\begin{itemize}[noitemsep,nolistsep,leftmargin=*]
    \item {阅读最新顶会论文与博客经验分享, 学习提升 BERT 和 ALBERT 模型在问答任务中效果的方案.}
    \item {通过在相似数据集上预训练和对困难样本的增强采样等训练技术, 用更小的模型超过 Kaggle 上的最优提交.}
    \item {完成全英文论文与项目汇报, 并在其他同学来自信息学院更高年级实验班的情况下, 获得最高的 GPA.\\}
\end{itemize}

%%%%%%%%%%%%%%%%%%%%%%%%%%%%%%%%%%%%%%%%%%%%%%%%%%%%%%%%%%%%%%%
% Extra Curricular Activities, Leadership, etc 
%%%%%%%%%%%%%%%%%%%%%%%%%%%%%%%%%%%%%%%%%%%%%%%%%%%%%%%%%%%%%%%
\noindent \textbf{\underline{\kaishu{部分荣誉}}} \\
\noindent \textbf{全国中学生物理竞赛(江西赛区)一等奖(实验部分全省第 8 名)} \hfill 2016.9 $-$ 2017.9 \\
\noindent \textbf{全国大学生数学建模竞赛(上海赛区)三等奖} \hfill 2020.9 $-$ 2020.9\\

%%%%%%%%%%%%%%%%%%%%%%%%%%%%%%%%%%%%%%%%%%%%%%%%%%%%%%%%%%%%%%%
% Other Skills: you can add all your other skills here.
% Continue to keep only relevant skills
%%%%%%%%%%%%%%%%%%%%%%%%%%%%%%%%%%%%%%%%%%%%%%%%%%%%%%%%%%%%%%%
\noindent \textbf{\underline{\kaishu{其他技能}}} \\
\noindent \textbf{语言能力:} 英语\textbf{(TOFEL 102 分)} \\
\noindent \textbf{担任职务:} 校乒乓球社团社长, 乒乓球队队长(曾获上海市锦标赛团体\textbf{第 3 名}, 混双\textbf{第 5 名})
\end{document}
