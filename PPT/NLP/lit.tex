\documentclass{beamer}
\usepackage{xeCJK,latexsym,amsmath,xcolor,multicol,booktabs,calligra}
% There are many different themes available for Beamer. A comprehensive
% list with examples is given here:
% http://deic.uab.es/~iblanes/beamer_gallery/index_by_theme.html
% You can uncomment the themes below if you would like to use a different
% one:
% \usetheme{AnnArbor}
% \usetheme{Antibes}
% \usetheme{Bergen}
% \usetheme{Berkeley}
% \usetheme{Berlin}
%\usetheme{Boadilla}
% \usetheme{boxes}
%\usetheme{CambridgeUS}
% \usetheme{Copenhagen}
% \usetheme{Darmstadt}
%\usetheme{default}
%\usetheme{Frankfurt}
%\usetheme{Goettingen}
%\usetheme{Hannover}
% \usetheme{Ilmenau}
%\usetheme{JuanLesPins}
% \usetheme{Luebeck}
\usetheme{Madrid}
%\usetheme{Malmoe}
% \usetheme{Marburg}
%\usetheme{Montpellier}
%\usetheme{PaloAlto}
%\usetheme{Pittsburgh}
%\usetheme{Rochester}
%\usetheme{Singapore}
% \usetheme{Szeged}
%\usetheme{Warsaw}
%\usetheme{Montpellier}
%\usecolortheme{beaver}
%\usecolortheme{lily}

\title{How I Overfitted}
% 打工人买房 有一套
% A subtitle is optional and this may be deleted

\author{石雨凌}

\institute{上海财经大学} % (optional, but mostly needed)
% - Use the \inst command only if there are several affiliations.
% - Keep it simple, no one is interested in your street address.

\date{2020.12.25}

% \subject{Theoretical Computer Science}

\AtBeginSection[]
{
  \begin{frame}<beamer>{Outline}
    \tableofcontents[currentsection, currentsubsection]
  \end{frame}
}

% Let's get started
\begin{document}

\begin{frame}
  \titlepage
\end{frame}

\begin{frame}{目录}
  \tableofcontents
  % You might wish to add the option [pausesections]
\end{frame}

\section{Pre-processing}

\begin{frame}{Pre-processing}
\begin{block}{观察与预处理}
\begin{itemize}
    \item 文本较为随意, 口语化
    \item 类别具有明显的长尾分布
    \item 去除一些无效据字段(如"市民反映", "望尽快解决", "此单来自谋某论坛" 等)
\end{itemize}

\end{block}

\end{frame}

\section{Model Selection}

\begin{frame}{Model Selection}
\begin{block}{Selected models}
\begin{itemize}
    \item ERNIE: mask 机制改进(如 knowlege masking), 加入贴吧语料
    \item RoBERTa-wwm-ext: whole word masking, 更多训练数据和迭代步数
    \item RoBERTa-large-pair: 对于句子对任务预训练
\end{itemize}

\end{block}
    \begin{figure}[htbp]
        \centering
        \includegraphics[width=0.4\textwidth]{figure/ernieandbert.jpg}
        \caption{ERNIE and BERT}
        \label{fig:my_label}
    \end{figure}
\end{frame}

\section{Pre Training}

\begin{frame}{Pre Training}
\begin{block}{相似任务预训练}
\begin{itemize}
    \item 借助相似文本分类任务\footnote{L. Pan, et al., 2019, Frustratingly Easy Natural Question Answering}: THUCNews 数据集中20万条新闻标题与新闻类别的分类 (2 epochs)
    \item Dev set 准确率提升 $0.3\%$ 左右
\end{itemize}

\end{block}
    \begin{figure}[htbp]
        \centering
        \includegraphics[width=1\textwidth]{figure/pretrain.png}
        \caption{模型准备}
        \label{fig:my_label}
    \end{figure}
\end{frame}

\section{Speed up}

\begin{frame}{Speed up}
    \begin{block}{混合精度训练}
    \begin{itemize}
        \item 利用 Apex\footnote{Micikevicius, et al., 2018. Mixed Precision Training.}, 梯度更新时采用单精度
        \item 显存减半, 速度加倍, 精度不减
    \end{itemize}
    \end{block}
    \begin{figure}
        \centering
        \includegraphics[width=0.7\textwidth]{figure/mixpre.png}
        \caption{Mixed precision training iteration for a layer.}
        \label{fig:my_label}
    \end{figure}
\end{frame}

\section{Loss Function}

\begin{frame}{Loss Function}
\begin{block}{Focal loss}
\begin{itemize}
    \item 面对样本不平衡问题, 减少易分类样本的损失\footnote{T.-Y. Lin, et al., 2017, Focal Loss for Dense Object Detection}
    \item Dev set 准确率提升 $0.5\%$ 左右 (LB: 0.798, 0.800)
\end{itemize}

\end{block}
    \begin{figure}[htbp]
        \centering
        \includegraphics[width=0.5\textwidth]{figure/focal.png}
        \caption{Loss Function}
        \label{fig:my_label}
    \end{figure}
\end{frame}

\section{Auxiliary Task}

\begin{frame}{Auxiliary Task}
\begin{block}{加入句子对任务}
\begin{itemize}
    \item 想法: labels 同样为中文文本, 可以视为 sentences 的摘要, 而 classification task 并未使用这一部分有价值的信息.
    \item 观察: 对 Dev set 上预测的 probabilities 进行排序, 正确标签 79\% 在第一位, 11 \% 在第二位, 选取至前5位可以覆盖 96\% 的正确标签
\end{itemize}
\end{block}
    \begin{figure}[htbp]
        \centering
        \includegraphics[width=0.9\textwidth]{figure/pairdata.png}
        \caption{Generating pair data}
        \label{fig:my_label}
    \end{figure}
\end{frame}

\begin{frame}{Auxiliary Task}
    \begin{block}{加入句子对任务}
\begin{itemize}
    \item 实施: 训练时仍以句子对任务分类情况计算交叉熵, 最终预测时选取句子对任务中 probability 最高者
    \item 效果: Dev set 准确率提升约 0.5\%
\end{itemize}
\end{block}

\begin{figure}[htbp]
    \centering
    \includegraphics[width=0.4\textwidth]{figure/pairbert.jpg}
    \caption{BERT for sentence pair tasks}
    \label{fig:my_label}
\end{figure}

\end{frame}

\section{Final Submission}

\begin{frame}{Final Submission}
\begin{block}{Results}
\begin{table}[htbp]
    \small
	\centering
	\caption{Experiment Results}
	\begin{tabular}{ccc}
		\toprule 
		Model & Public score & Private score \\
		\midrule 
		ERNIE\footnote{Original Task: text classification} & 0.7981 & 0.8000 \\
		ERNIE\footnote{Auxiliary Task: sentence pair classification} & 0.8042 & 0.8010 \\
		RoBERTa-large-pair${}^{b}$ & 0.8049 & 0.7979 \\
		RoBERTa-wwm-ext${}^{b}$ & 0.8021 & 0.7970 \\
		Ensemble${}^{b}$ & 0.8044 & 0.8008 \\
		\bottomrule 
	\end{tabular}
\end{table}
\end{block}
\end{frame}

\begin{frame}{想法}
\begin{itemize}
    \item Pseudo labels: (full, top8000, top4000) 但是效果并不明显 (top4000 +0.2\%), 最终未使用.
    \item 回译: 效果也不明显, 可能是句子相对口语化.
    \item BERT + CNN/RNN: 需要更多epoch, 效果提升 0.1\%, 但分布情况更加分散(top5 95\%), 最终未使用.
    \item 辅助任务设计数据量大(每个epoch约20min), 不便调参和加入 cross validation, 导致模型泛化性能一般.
\end{itemize}
    
\end{frame}

% \begin{frame}{Final Submission}
% \begin{block}{加入句子对任务}
% \begin{itemize}
%     \item 
% \end{itemize}

% \end{block}
%     \begin{figure}[htbp]
%         \centering
%         \includegraphics[width=0.35\textwidth]{figure/pairbert.jpg}
%         \caption{BERT for sentence pair}
%         \label{fig:my_label}
%     \end{figure}
% \end{frame}

\begin{frame}
    \begin{center}
        {\Huge\calligra Thanks!}
    \end{center}
\end{frame}

% \begin{frame}
%     \begin{center}
%         {\Huge Q $\&$ A}
%     \end{center}
% \end{frame}

\end{document}


